\documentclass[11pt,a4paper]{article} 
2 
 
3 \usepackage{amsmath,amssymb,amsfonts,verbatim} 
4 \usepackage[margin=1.5cm, vmargin=2cm]{geometry} 
5 \usepackage{graphicx} 
6 \usepackage{xcolor} 
7 \usepackage{commath} 
8 \usepackage{hyperref} 
9 \usepackage{float} 
10 \usepackage{array} 
11 
 
12 \def\N{\mathbb{N}} 
13 \def \Z {\mathbb{Z}} 
14 \def \Q {\mathbb{Q}} 
15 \def \R {\mathbb{R}} 
16 
 
17 \usepackage{fancyhdr}  
18 \pagestyle{fancy} 
19 \fancyhead{}          
20 \fancyhead[C]{Quantifying Flooding and the Effectiveness of Flood Mitigation Schemes of Rivers in the North of the UK}  
21 \fancyhead[L]{MATH 3001}      
22 \fancyhead[R]{2018/19}  
23 
 
24 \begin{document} 
25 
 
26 \setlength{\parindent}{0cm} 
27 
 
28 \begin{titlepage} 
29 \begin{center} 
30 Antonia Feilden\\ 
31 Student ID: 200989495\\ 
32 \vspace{2cm} 
33 {\huge \textbf{MATH3001: Project in Mathematics}}\\ 
34 \hrulefill 
35 
 
36 \vspace{1cm} 
37 {\LARGE Quantifying Flooding and the Effectiveness of Flood Mitigation Schemes of Rivers in the North of the UK}\\ 
38 \vfill 
39 \end{center} 
40 \end{titlepage} 
41 
 
42 \tableofcontents  
43 \noindent \hrulefill 
44 
 
45 \newpage 
\section{Introduction}

A flood is defined as “a great flowing or overflowing of water, especially over land not usually submerged.”\cite{1}

The frequency of extreme flooding in the UK has been predicted to increase with climate change. 17,000 properties were affected by flooding in the winter 2015-2016 in the North of the UK. December 2015 has been reported as the wettest month ever recorded.\cite{2} In 2017, flooding was a natural hazard identified as a major threat to the UK on the UK’s Risk Register.\cite{3} 

There are 4 main categories of flooding: fluvial (caused by a river bursting its banks), pluvial (caused by surface water run-off), coastal (caused by extreme tidal conditions) and reservoir (caused by dam failure).\cite{4}
This project studies river floods. Fluvial, pluvial and reservoir flooding are in scope of this project whilst coastal flooding is out of scope.

\begin{thebibliography}{}
\bibitem{1}Dictionary. Definition of “Flood”. [Online]. 2019. [Accessed 2 January 2019]. Available from: https://www.dictionary.com/browse/flood
\bibitem{2}Environmental Agency. Climate change means more frequent flooding, warns Environment Agency.[Online].2018.[Accessed 2 January 2019]. Available from: https://www.gov.uk/government/news/climate-change-means-more-frequent-flooding-warns-environment-agency
\bibitem{3}UK Government. National Risk Register Of Civil Emergencies 2017 Edition. [Online].2017.[Accessed 2 January 2019]. Available from: https://assets.publishing.service.gov.uk/government/uploads/system/uploads/attachment_data/file/644968/UK_National_Risk_Register_2017.pdf
\bibitem{4}Ambiental. Types of flood and flooding impact.[Accessed 2 January 2019]. Available from: https://www.ambiental.co.uk/types-of-flood-and-flooding-impact/



